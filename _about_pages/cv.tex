\documentclass[letterpaper,11pt]{article}
\newlength{\outerbordwidth}
\pagestyle{empty}
\raggedbottom
\raggedright
\usepackage[svgnames]{xcolor}
\usepackage{framed}
\usepackage{times}
\usepackage{tocloft}
\usepackage{graphicx}
\usepackage{multirow}
\usepackage[utf8]{inputenc}
\usepackage{tabularx}
\usepackage{textcomp}
\usepackage{ulem}
\newcommand{\li}{\uline{\hspace{0.5em}}}
\title{Cheng-CV}
%-----------------------------------------------------------
%Edit these values as you see fit

\setlength{\outerbordwidth}{3pt}  % Width of border outside of title bars
\definecolor{shadecolor}{gray}{0.75}  % Outer background color of title bars (0 = black, 1 = white)
\definecolor{shadecolorB}{gray}{0.93}  % Inner background color of title bars


%-----------------------------------------------------------
%Margin setup

\setlength{\evensidemargin}{-0.25in}
\setlength{\headheight}{0in}
\setlength{\headsep}{0in}
\setlength{\oddsidemargin}{-0.25in}
\setlength{\paperheight}{11in}
\setlength{\paperwidth}{8.5in}
\setlength{\tabcolsep}{0in}
\setlength{\textheight}{9.5in}
\setlength{\textwidth}{7in}
\setlength{\topmargin}{-0.3in}
\setlength{\topskip}{0in}
\setlength{\voffset}{0.1in}


%-----------------------------------------------------------
%Custom commands
\newcommand{\resitem}[1]{\item #1 \vspace{-2pt}}
\newcommand{\resheading}[1]{\vspace{8pt}
  \parbox{\textwidth}{\setlength{\FrameSep}{\outerbordwidth}
    \begin{shaded}
\setlength{\fboxsep}{0pt}\framebox[\textwidth][l]{\setlength{\fboxsep}{4pt}\fcolorbox{shadecolorB}{shadecolorB}{\textbf{\sffamily{\mbox{~}\makebox[6.762in][l]{\large #1} \vphantom{p\^{E}}}}}}
    \end{shaded}
  }\vspace{-5pt}
}
\newcommand{\ressubheading}[4]{
\begin{tabular*}{6.5in}{l@{\cftdotfill{\cftsecdotsep}\extracolsep{\fill}}r}
		\textbf{#1} & #2 \\
		\textit{#3} & \textit{#4} \\
\end{tabular*}\vspace{-6pt}}
%-----------------------------------------------------------


\begin{document}

%-----------------------------------------------------------
%Insert IIT Madras Logo 
\begin{tabular*}{7.1in}{l@{\extracolsep{\fill}}r}
  & \multirow{8}{*}{\includegraphics[scale=0.21]{collegelogo.jpg}}\\
  & \\
%-----------------------------------------------------------  
  \textbf{\Huge Zhan Cheng} & \\
  \\
  \large https://enchanted23.github.io & \\
  \large cz23@mail.ustc.edu.cn \\
\end{tabular*}
\\


%%%%%%%%%%%%%%%%%%%%%%%%%%%%%%
\resheading{Education}
%%%%%%%%%%%%%%%%%%%%%%%%%%%%%%
\begin{itemize}
\item
	\ressubheading{\Large UNIVERSITY OF SCIENCE AND TECHNOLOGY OF CHINA}{Hefei, China}{Undergraduate}{2014 - now}
	\begin{itemize}
    		\resitem{\textbf{Major: } \ Computer Science and Technology}
	\end{itemize}
\end{itemize}

%%%%%%%%%%%%%%%%%%%%%%%%%%%%%%
\resheading{Research Experiences}
%%%%%%%%%%%%%%%%%%%%%%%%%%%%%%
\begin{itemize}
\item
	\ressubheading{\Large English Chatbot Completion Using HRED Model}{University of Waterloo}{Summer Researcher}{Jun. 2017 - Aug. 2017}
	\begin{itemize}
		\resitem{\textbf{Mentor: } \ {Prof. Ming Li} \ $|$ \ https://cs.uwaterloo.ca/\texttildelow mli/}
		\resitem{\textbf{Description: } 
		\\ 1. Built and improved the Hierarchical Recurrent Encoder-Decoder (HRED) \\model for English chatbot referring to some papers and code. 
		\\ 2. Collected tens of millions of data from different sources like twitter, \\reddit, movie dialog, etc to train the RNN model.
		\\ 3. Modified the code and the trainning sets repeatedly according to the \\performance of the chatbot until it achieved a desired result.}
	\end{itemize}

\item
	\ressubheading{\Large Improving Humanoid Robot Jia Jia}{University of Science and Technology of China}{Undergraduate Researcher}{July 2016 - now}
	\begin{itemize}
		\resitem{\textbf{Mentor: } \ Prof. Xiaoping Chen \ $|$ \ http://ai.ustc.edu.cn/en/people/xpchen.php}
		\resitem{\textbf{Description: } 
		\\ 1. Enhancing the robot's Chinese Conversation System to improve its \\performance of
communicating with human beings.
		\\ 2. Applied machine learning algorithms to the Question Classification \\Model of Jia Jia.
		\\ 3. Doing some tasks like building a WeChat version of Jia Jia, labeling data, etc.}
	\end{itemize}

\item
	\ressubheading{\Large Knowledge Base Completion Task}{University of Science and Technology of China}{Undergraduate Researcher}{Sept. 2016 - now}
	\begin{itemize}
		\resitem{\textbf{Advisor: } \ Doctor Yi Zhou \ $|$ \ Y.Zhou@westernsydney.edu.au}
		\resitem{\textbf{Description: } 
		\\ 1. Analyzed traditional Knowledge Bases' deficiencies in solving \\Winograd Schema Challenge.
		\\ 2. Building and improving a Automated Knowledge Base \\Construction System referring to ProBase and DeepDive.
		\\ 3. Writing a paper about the things listed above.}
	\end{itemize}
\end{itemize}


%%%%%%%%%%%%%%%%%%%%%%%%%%%%%%
\resheading{Research Interests}
%%%%%%%%%%%%%%%%%%%%%%%%%%%%%%
	\vspace{-2pt}
	\begin{center}\begin{tabular*}{6.6in}{l@{\extracolsep{\fill}}r}
		\multicolumn{2}{l}{{\bf Artificial Intelligence: } \ Deep Learning, \ Natural Language Processing, \ Human-Robot Interaction}\\
		\multicolumn{2}{l}{ \qquad \qquad \qquad \qquad \qquad \ Machine Learning and Data Science, \ Knowledge Graph Construction}\\
		\multicolumn{2}{l}{{\bf The others: } \ Virtual Reality, \ Autonomous Vehicles, \ Social Network}\\
		\vphantom{E}
\end{tabular*}
\end{center}\vspace*{-16pt}


%%%%%%%%%%%%%%%%%%%%%%%%%%%%%%
\resheading{Skills}
%%%%%%%%%%%%%%%%%%%%%%%%%%%%%%
\begin{itemize}
\item
	Programming and Markup Languages
	\begin{itemize}
		\resitem{{\bf Adept:} Python, Java, C}
		\resitem{{\bf Intermediate:} MATLAB, HTML, JavaScript, \LaTeX}
	\end{itemize}

\item
	Tasks
	\begin{itemize}
		\resitem{{\bf Adept:} TensorFlow Programming, Web Crawling, Dialogue System Design}
		\resitem{{\bf Intermediate:} Visual Software Design, Web Design, Mobile Application Development}
	\end{itemize}

\end{itemize}

\end{document}